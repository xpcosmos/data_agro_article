

\section{Material e métodos}


\subsection{Fonte de dados}

As pesquisas usarão de fontes de dados públicas para a criação de banco de dados para análise das variáveis e relacioná-las sob a perspectiva
econômica, social e ambiental, a saber: IBGE (Instituto Brasileiro de Geografia e Estatística), SIDRA-IBGE (Sistema IBGE de Recuperação Automática), Mapbiomas Brasil. É possível verificar as variáveis estudadas nas tabelas \ref{tab:produto_analisados-lavtemp} e \ref{tab:produto_analisados-pecuaria}.

\begin{center}
    \begin{table}[!htb]
        \centering
        \begin{tabular}{|p{0.8\textwidth}|}
        \hline
        \textbf{Produtos das Lavouras temporárias} \\ \hline
           Abacaxi \\ 
            \hline
           Batata-doce \\ 
            \hline
           Cebola \\ 
            \hline
           Feijão \\
            \hline
           Melancia \\
            \hline
           Milho \\
            \hline
           Tomate \\
            \hline  
            \end{tabular}
        \caption{Produtos analisados das produções das lavouras temporárias}
        \label{tab:produto_analisados-lavtemp}
    \end{table}
\end{center}


\begin{center}
    \begin{table}[!htb]
        \centering
        \begin{tabular}{|p{0.8\textwidth}|}
        \hline
        \textbf{Produtos da Agropecuária} \\ 
            \hline  
            Bovino \\
            \hline  
            Equino \\
            \hline  
            Caprino \\
            \hline  
            Ovino \\
            \hline  
            \end{tabular}
        \caption{Produtos analisados das produções da pecuária}
        \label{tab:produto_analisados-pecuaria}
    \end{table}
\end{center}


\subsection{Processamento de dados}

O processamento dos dados e análise ocorreram na plataforma de mineração de dados Orange, através da linguagem de programação Python e suas bibliotecas de aprendizado de máquina e ciência de dados, e na plataforma gratuita Google Earth Engine para a investigação de imagens de satélite. Buscar-se-a dados públicos com o objetivo de predizer as tendências de crescimento, estabilização ou redução, a saber: em hectares,
Pastagem, Soja, Lavouras Temporárias (junção de milho e abacaxi e outras lavouras temporárias); em valores quantitativos. Os dados coletados serão dados anuais, num intervalo entre 1985 e 2021, somando 36 anos de análise. É possível ver um exemplo dos dados resultantes na tabela \ref{tab:sample10}

\begin{table}[hbt!]
    \centering
    \resizebox{\textwidth}{!}{\begin{tabular}{lrlrrllrlrlrlrr}
\toprule
                  cidade &  cod\_cidade &  nome\_rgi &  cod\_rgi &  cod\_rgint & nome\_rgint &                    tipo\_producao &  cod\_variavel &             variavel &  cod\_produto &             produto &  cod\_metrica &   metrica &  valor &    ano \\
\midrule
    Santa Izabel do Pará &     1506500 &     Belém &   150001 &       1501 &      Belém & Produto das lavouras temporárias &         214.0 & Quantidade produzida &       2694.0 &         Batata-doce &       1017.0 & Toneladas &    0.0 & 1987.0 \\
     Santana do Araguaia &     1506708 &  Redenção &   150012 &       1504 &   Redenção & Produto das lavouras temporárias &         214.0 & Quantidade produzida &       2706.0 &       Malva (fibra) &       1017.0 & Toneladas &    0.0 & 2001.0 \\
       Concórdia do Pará &     1502756 &     Belém &   150001 &       1501 &      Belém & Produto das lavouras temporárias &         214.0 & Quantidade produzida &       2706.0 &       Malva (fibra) &       1017.0 & Toneladas &  120.0 & 1994.0 \\
Brejo Grande do Araguaia &     1501758 &    Marabá &   150009 &       1503 &     Marabá & Produto das lavouras temporárias &         109.0 &        Área plantada &       2714.0 &     Sorgo (em grão) &       1006.0 &  Hectares &    0.0 & 2012.0 \\
São Domingos do Araguaia &     1507151 &    Marabá &   150009 &       1503 &     Marabá & Produto das lavouras temporárias &         214.0 & Quantidade produzida &       2709.0 &            Melancia &       1017.0 & Toneladas &    0.0 & 2018.0 \\
                   Bagre &     1501105 &    Breves &   150020 &       1507 &     Breves & Produto das lavouras temporárias &         214.0 & Quantidade produzida &       2691.0 & Amendoim (em casca) &       1017.0 & Toneladas &    0.0 & 1981.0 \\
                 Piçarra &     1505635 &    Marabá &   150009 &       1503 &     Marabá & Produto das lavouras temporárias &         216.0 &         Área colhida &       2706.0 &       Malva (fibra) &       1006.0 &  Hectares &    0.0 & 1982.0 \\

\bottomrule

\end{tabular}}
    \caption{Amostra com dez registros aleatórios extraídos da tabela} 
    \label{tab:sample10}
\end{table}

Inicialmente foi necessário tratar e integrar os dados das diferentes bases formando um conjunto de dados únicos. Após isso, os dados foram modelados para se adequarem a finalidade requerida. Além da integração das duas bases, foi incluído também detalhamentos geográficos aos dados, como regiões intermediárias e imediatas.



% --------------------

% --------------------
\subsubsection{Dados de cobertura da terra}

O MapBiomas é uma iniciativa sem fins lucrativos do Sistema de Estimativas de Emissões de Gases de Efeito Estufa do Observatório do Clima mantido e desenvolvido por várias instituições, entre elas, ONGs e Universidades Federais. Seus dados são disponibilizados na plataforma gratuitamente que também disponibiliza tutoriais para baixar os arquivos. 

A sua utilização no presente projeto se dá pela possibilidade de adquirição dos dados de forma mais atualizada, veloz e segundo os próprios integrantes do projeto, mais barata, em comparação com outros métodos que almejam a mesma finalidade. Os dados de cobertura de terra do MapBiomas foram extraídos em sua totalidade, entretanto, para finalidade de alinhamento das informações dispostas com o objetivo do projeto, os dados foram filtrados para incluir somente os registros do estado do Pará. 

Após a limpeza, foi contabilizado registros de 143 diferentes municípios do estado do Pará. Um dos atributos disponibilizados na base é referente ao bioma, entretanto, por se limitar apenas a duas categorias distintas, foi desconsiderada. As informações preservadas se reservaram ao município, identificador de classe, nome da classe e ano do registro. As classes de interesse foram mantidas e as demais foram descartadas, sendo classes de interesse: Formação florestal, pastagem, grãos de soja e outras lavouras temporárias. Os dados foram então armazenados e reservados para uma posterior integração com os demais dados.

\subsubsection{Dados do IBGE}

% O que

O SIDRA (Sistema IBGE De Recuperação Automática) é um banco de dados agregados que disponibiliza séries temporais de diversas áreas, como: Administração pública, agroindústria, economia, economia agrícola e estatística.
% Por que?

Essa ferramenta possibilita a recuperação de dados diretamente do sistemas de dados do IBGE. Dentro do sistema, diversas tabelas são disponibilizadas com as mais diferentes finalidades. As tabelas extraídas foram: Área plantada, área colhida, quantidade produzida, rendimento médio e valor da produção das lavouras temporárias; Efetivo dos rebanhos, por tipo de rebanho. Sua maior relevância para o projeto é em virtude da confiabilidade dos dados, considerando que são fornecidos diretamente pelo maior órgão oficial de estatística do Brasil.
% Como?

Os dados podem ser adquiridos através de API. Os registros possuíam algumas características que ocasionariam erros, visto isso, os dados passaram por transformações e limpezas que garantiram a qualidade e coerência dos dados. Após essa limpeza, as informações foram mescladas com os dados de cobertura da terra e realizado uma cuidadosa análise. 



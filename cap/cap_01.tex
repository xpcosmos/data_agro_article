
\section{Introdução}


Muitas das políticas agrícolas para a Amazônia brasileira tiveram implantações no Estado do Pará, seja pelas condições edafoclimáticas, seja pelas condições logísticas que tornaram o território paraense parte fundamental da fronteira agrícola brasileira \cite{maeda2009predicting}. O Pará detém algumas trajetórias produtivas de destaque na agropecuária brasileira \cite{pespectiva}, a exemplo da produção abacaxi, soja e produtos da pecuária as quais são as principais do Brasil; no que se refere a pecuária, o Pará é o quarto em rebanho de bovinos e equinos (IBGE, 2021).

Convêm destacar que as atividades rurais detêm efeitos a jusante e a montante da cadeia produtiva, isto é, incorporam elementos fornecedores de insumos e intermediários dos bens finais \cite{pinage2022forest}. Além de propiciar um ambiente de negócios crescente e dinâmico, em especial, quando se considera a geração de empregos e a criação de empresas no ramo agropecuário, inclusive formais, que colaboram tanto na arrecadação de impostos, quanto no desempenho econômico estadual e do país (BRASIL, 2021).

Neste contexto, novas tecnologias têm se destacado nos mercados produtivos, em especial, no agronegócio. Tudo para facilitar a gestão, com o objetivo de diminuir o tempo e o custo, e aumentar a produtividade e a sustentabilidade \cite{renzcherchen2021desenvolvimento}. São desde aplicativos para promover a previsibilidade climática e classificação do uso da terra com base em imagens de satélites e drones, desde softwares de gerenciamento das atividades da fazenda, a sistemas de acompanhamento das oscilações de preços do mercado \cite{gardon2020brazil}.

Atualmente, as tecnologias de Inteligência Artificial atuam em uma grande quantidade de dados, fornecendo diagnósticos e previsibilidade, projetando cenários, antecipando situações indesejáveis e fazendo recomendações em tempo real \cite{de2020mineraccao}. Em consonância, o uso da geo informação, ciência de dados e análise de séries temporais também são ferramentas que permitem aprofundar o conhecimento e estabelecer relações entre as mais diversas variáveis. Ajudam essencialmente a compreender as dimensões espaciais e temporais dos dados, ao nível de identificar quais dados devem ser coletados, com que frequência e em quais áreas determinadas

Em contrapartida, nem todos os dados desejados estão disponíveis, e mesmo se existirem, muitas vezes, deverão ser remodelados até alcançarem um grau de relevância satisfatório. Paralelamente, existe uma infinidade de técnicas que podem contribuir com a projeção de séries temporais e o entendimento histórico dos dados. Além disso, base de dados públicas raramente encontram-se organizadas, sem qualquer integração, o que dificulta o uso de certos algoritmos de modo a gerar dados confiáveis e mais relevantes para o tomador de decisão, seja público ou privado.

O objetivo desse trabalho é realizar a modelagem de dados e a projeção da trajetória da pecuária e lavouras temporárias do estado do Pará, suas regiões e municípios. Essa modelagem de dados é possível usando os métodos computacionais de mineração de dados e inteligencia artificial que auxiliam na identificação de tendências das atividades agropecuárias do estado. 

Assim, o estudo desses dados e técnicas avançadas se torna fundamental para o planejamento e a definição de estratégias, seja para aproveitar o desempenho produtivo do setor agropecuário, com o atendimento da demanda de mão-de-obra ou oferta de produtos e serviços, seja para conter alguma externalidade, como a degradação ambiental. O repositório desse projeto se encontra no link: \url{https://github.com/npcateam2/data_agro}